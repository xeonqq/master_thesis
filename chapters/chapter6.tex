\chapter{Summary\label{cha:chapter6}}
\section{Discussion}\label{sec:dis}
\glspl{GPGPU} as the name suggests could be used for general purpose computation and isn't tightly bound to the application. \glspl{GPU} could be used as a platform to run several different applications at different instants of time. The software programmability of the \glspl{GPU} offers a high degree of flexibility in terms of application adaptation. \glspl{GPU} also offer good backward compatibility. If an algorithm changes, the new software could run on older chip sets. The current \gls{GPU} technologies have not proven to be space-qualified and radiation-tolerant. But, \glspl{GPU} could easily be used in low Earth orbits and aeronautics.

\subsection{Fault-tolerance for applications}\label{sec:dis:fault}
The suitability of architectures in space is very important as it has direct bearing on the outcome of the algorithms. Architectural abnormalities could result in bit flipping to cause erroneous or incorrect computations. The abnormalities might be unpredictable as in, the number of times the hardware is being faulty. Even if there is some sort of reliability for the hardware to be fault tolerant to a specified percentage, it still would be not considered to be space qualified as the erroneous behaviour totally depends on the nature of the applications under consideration. For scientific missions especially, the data is required to be compressed in lossless mode without any errors so that it could be transmitted back to the \gls{GCS} to decode it and analyse. In such cases, the architectural anomalies are totally unacceptable. However in low earth orbits and earth observation applications, the errors in the application could be tolerable if few bits are incorrect owing to the faulty computing architecture. Hence it is really important for the architectures to be fool-proof in their operation in order to be suitable for space applications in general.

\section{Dissemination}\label{sec:diss}
Who uses your component or who will use it? Is it integrated into a larger environment? Did you publish any papers?

\section{Conclusion}\label{sec:conc}
The high-resolution satellite imaging systems require real-time image compressors on-board due to limited storage to store uncompressed raw data and/or also to conserve communication bandwidths. Hence it is mandatory for the image compressors to guarantee high compression throughput in the order of several \unitfrac[]{Mpx}{s}. \glspl{GPU} owing to their massive number of computing cores is investigated as a potential architecture platform to run the CCSDS 122.0-B-1 image data compression standard in lossless mode. The \gls{BPE} stage of the standard was parallelized on \gls{GPU} to satisfy the high compression throughput requirements. The parallelized \gls{BPE} achieved an average speed-up of $16.718$ times the host CPU implementation. The \gls{GPGPU} solution is approximately $2.59$ times slower in comparison to the state of the art hardware FPGA solution. This paper explores the possibilities of usage of \gls{GPU} technologies for space applications provided they become radiation-tolerant and space-qualified in the future.
