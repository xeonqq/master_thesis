\thispagestyle{empty}
\chapter*{Abstract}

Self-localization is a crucial part for autonomous robot, particularly in the RoboCup \gls{SPL}. It is the prerequisite for the robot to accomplish the remaining decision making tasks such as passing the ball to the teammates, saving the goal, path planning and so on. In short, the robot needs to know where it is in the game field. The challenge herein is mainly to estimate the robot's position accurately. Since the robot only observes its surrounding world partially at a given time, it is sometimes not sufficient enough to determine its exact position. In addition, the robot must also identify the landmarks on the game field which could be ambiguous in most of the cases. 
%lies in the domain of localization is that robots have only partial observations about the surrounding world, they need to fuse sensor information to estimate its location. 
%Many localization algorithms have been proposed in the past. Most of them are variants based on Bayesian theories, and the two widely used algorithms are particle filter and Kalman filter. Although, the particle filter algorithm is robust in the sense of mitigating the false positives and recovering robot from the de-localized situation, the Kalman filter algorithm, on the other hand outperforms the former in terms of localization accuracy, and offers smooth walking trails for the robot. 

In this thesis, two localization methods are investigated for NAO robot in the RoboCup \gls{SPL}, \ie{} optimization based localization and multi-hypotheses Kalman filter localization.
Optimization based localization differs the classic Kalman filter in the measurement update step, where the optimization based localization estimates the system state by minimizing the measurement error of field line points in a iterative optimization process. 
On the other hand, given more available landmark features, multi-hypotheses Kalman filter localization describes a multi-modal probability distribution and can fast re-localize kidnapped robot by incorporating the sensor resetting technique from particle filter. At the same time, it also inherits the advantage from Kalman filter of being accurate and efficient in position estimation. In the benchmark, this thesis shows multi-hypotheses Kalman filter localization outperforms optimization based localization and particle filter localization in terms of accuracy and efficiency.
