\thispagestyle{empty}
\vspace*{1.0cm}

\begin{center}
    \textbf{Abstract}
\end{center}

\vspace*{0.5cm}

\noindent

%The usage of \glspl{GPU} as computing architectures for inherently data parallel signal processing applications in this computing era is very popular. In principle, \glspl{GPU} in comparison with \glspl{CPU} could achieve significant speed-up over the latter, especially considering the data parallel applications which expect high throughput. The thesis investigates the usage of \glspl{GPU} for running space borne image data compression algorithms, in particular the CCSDS 122.0-B-1 standard as a case study. The thesis proposes an architecture to parallelize the \gls{BPE} stage of the CCSDS 122.0-B-1 in lossless mode using a \gls{GPU} to achieve high throughput performance to facilitate real-time compression of satellite image data streams. Experimental results are furnished by comparing the performance in terms of compression time of the \gls{GPU} implementation versus a state of the art single threaded \gls{CPU} and an \gls{FPGA} implementation. The \gls{GPU} implementation on a NVIDIA{\textregistered} GeForce{\textregistered} GTX 670 achieves a peak throughput performance of \unitfrac[162.382]{Mbyte}{s} (\unitfrac[932.288]{Mbit}{s}) and an average speed-up of at least 15 times the software implementation running on a \unit[3.47]{GHz} single core Intel{\textregistered} Xeon{\texttrademark} processor. The high throughput CUDA implementation using \glspl{GPU} could potentially be suitable for air borne and space borne applications in the future, if the \gls{GPU} technology evolves to become radiation-tolerant and space-qualified.

Self-localization is a crucial part for autonomous robot, particularly in the RoboCup \gls{SPL}. It is the prerequisite for the robot to accomplish the remaining decision making tasks. In short, the robot needs to know where it is in the game field.  The challenge lies in the domain of localization is that robots have limited information about the surrounding world, they need to fuse sensor information to estimate its location. Many localization algorithms have been proposed in the past years. Most of them are variants based upon Bayesian theories, the two widely used algorithms are particle filters and multi-model Kalman filters. Although particle filters are robust, multi-model Kalman filters outperforms it in terms of accuracy. In this thesis, an efficient localization algorithm will be investigated by combining the essence of particle filter and the Kalman filter to gain better performance for localization and implemented for NAO robot in the RoboCup \gls{SPL}.
