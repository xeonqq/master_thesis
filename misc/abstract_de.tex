\thispagestyle{empty}
\vspace*{0.2cm}

\begin{center}
   \LARGE \textbf{Zusammenfassung}
\end{center}

\vspace*{0.5cm}

\noindent 
Selbstlokalisierung ist ein entscheidender Bestandteil f�r autonome Roboter, insbesondere in der RoboCup
\gls{SPL}. 
Es ist die Voraussetzung f�r den Roboter zu erreichen, die verbleibende Entscheidungsaufgaben. 
Kurz gesagt, zu wissen, wo es ben�tigt der Roboter in dem Spielfeld. 
Die Herausforderung liegt im Bereich der Lokalisierung ist, dass Roboter begrenzte Informationen �ber die Au�enwelt, um Sensorinformationen verschmelzen m�ssen sie seiner Lage zu sch�tzen. 
Viele Lokalisierungsalgorithmen wurden in der Vergangenheit vorgeschlagen worden Jahren. Die meisten von ihnen sind Varianten, basierend auf Bayesian Theorien, die zwei weit verbreitete
Algorithmen sind Partikelfilter und Multi-Modell-Kalman-Filter. Obwohl Partikelfilter
sind robust, Multi-Modell-Kalman-Filtern �bertrifft es in Bezug auf Genauigkeit. 
In diesem Diplomarbeit, wird eine effiziente Lokalisierungsalgorithmus durch die Kombination der Essenz untersucht werden der Partikelfilter und das Kalman-Filter, um eine bessere Leistung f�r die Lokalisierung zu gewinnen und
f�r NAO-Roboter im RoboCup SPL realisiert.
